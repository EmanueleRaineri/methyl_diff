\documentclass[11pt]{amsart}
\usepackage{eulervm}
\renewcommand{\rmdefault}{pplx}
\title{Computing exact differences between beta distribution in genomic studies}
\author{ER, MDC,SH}
\date{\today}
\begin{document}
\begin{abstract}
We apply a known algorithm for computing the exact difference between beta distributions
to compute the probability that a given position in a genome is differentially methylated across
individuals. We discuss the advantages brought by the adoption of the exact solution
with respect to two simpler approximation (Fisher's test and Z score).
The same formalism can be applied in a similar way to variant calling.
\end{abstract}
\maketitle
\section{Introduction}
\subsection{Beta distribution to model methylation probabilities}
The Beta probability distribution (defined explicitly below)  appears very naturally in many analyses of genomic data : tipically such analysis also entail the comparison between different samples, which in turn means that different Beta pdfs have to be combined. Here I describe a software which computes the difference between two Beta distributions, and show which difference there is between the exact computation and two sensible approximations to it, namely performing a Fisher's test on the counts and computing a Z score (thereby replacing the Beta with a Gaussian). For concreteness I will describe the case of measuring DNA methylation through whole genome bisulfite sequencing (WGBS in what follows), but the same formalism would apply with almost no change to SNP calling. Let us first fix some notation. I will indicate the Beta probability distribution (over $\theta$) with parameters $\alpha,\beta$ with $\mbox{Beta}(\alpha,\beta)$. On the other hand, I'll use the letter $B$ for the Beta function 
\[B(\alpha,\beta)=\frac{(\alpha-1)!(\beta-1)!}{(\alpha+\beta-1)!}\]. The function plays a role in the definition of the distribution \[\mbox{Beta}(\alpha,\beta)=\frac{\theta^{\alpha-1}(1-\theta)^{\beta-1}}{B(\alpha,\beta)}\]
 
Consider a set of reads out of a WGBS experiment covering a certain genomic coordinate $x$. Since not all cells in the sample being sequenced will, in general,  have the same basis methylated at the same time, the final result of a bisulfite experiment will be a collection of heterogeneous reads : some will indicate methylation at position $x$ (these are the so called {\em non converted} reads), others (the {\em converted} reads) will correspond to samples that are not methylated at position $x$. What one wants to estimate is the probability $\theta$ of methylation i.e. the fraction of bases methylated at position $x$.  If $\theta$ were known a priori, the probability of obtaining $n$ non converted reads is :
\[P(n|\theta)={d \choose n}\theta^n ({1-\theta})^(d-n)=\frac{1}{d+1}\mbox{Beta}(n+1,d-n+1)\]
If one assumes a uniform prior on $\theta$, $P(\theta)=1 \ \forall \theta \in [0,1]$ the expression for $P(\theta|n)$ is very similar \footnote{The factor $\frac{1}{d+1}$ cancels out when applying Bayes' theorem}
\[P(\theta|n)=\mbox{Beta}(n+1,d-n+1)\].
let us considers a genomic position in two individuals, and the corresponding methylation probabilities $\theta_1,\theta_2$. 
To explore whether the same position is differentially methylated across the individuals with unconverted reads respectively $n_1,n_2$ and read depths $d_1,d_2$ one
has to compute $P(\theta_1>\theta_2)$ that is \[P(\mbox{Beta}(n_1+1,d_1-n_1+1)>\mbox{Beta}(n_2+1,d_2-n_2+1))\]
 
This problem can be tackled approximately :  for example performing a Fisher test over a contingency table built with the number of non converted and converted reads in the two samples. Another quick approximation consists in computing a Z-score, which is usually transformed into a $p$-value assuming that $\theta_1,\theta_2$ are Gaussian (with the same mean and variance as the underlying Beta distribution).  

\subsection{Exact computation of beta differences}



\section{Results} 
Below I show the difference between adopting these approximation and using an exact computation.
\section{Methods}
I organized the comparison between the exact and approximate solution in two steps. First,
I looked at the behaviour of the two tests on a pair of real samples. The results are shown in \ref{cmpreal} : the left part of the figure  I generated a pair of samples whose counts are generated by the same underlying binomial process at different coverages, i.e. $\theta_1=\theta_2=0.5$. These consitute a negative control, in the sense that all the different methods should not report significant differences between the samples. Furthermore, I generated a pair of samples such that their underlying binomial probabilities are markedly different $\theta_1=0.9,\theta_2=0.5$; those are the true positives, samples for which the test should detect that $\theta_1>\theta_2$. I can then compare the ROC curves of the three methods for different values of the sample coverages, $d_1,d_2$. 

\end{document}
