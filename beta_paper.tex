\documentclass{amsart}
\usepackage{eulervm}
\renewcommand{\rmdefault}{pplx}
\title{Computing exact differences between beta distribution in genomic studies}
\author{ER, MDC,SH}
\begin{document}
\begin{abstract}
We apply a known algorithm for computing the exact difference between beta distributions
to compute the probability that a given position in a genome is differentially methylated across
individuals. We discuss the advantages brought by the adoption of the exact solution
with respect to two simpler approximation (Fisher's test and Z score).
The same formalism can be applied to variant calling.
\end{abstract}
\maketitle

\section{Beta distribution to model methylation probabilities}
Given a population of cells, not all of them will be methylated at a given
genomic position (especially if one considers non CpG methylation).
If the probability for a position to be methylated is $p$, and one performs whole genome
bisulfite sequencing, one can expect that the number of non converted reads  will be:
${C \choose n}n^p(C-n)^{1-p}$.
If, on the other hand, one is given the number of non converted reads and wants to estimate the parameter $p$, one finds that it follows a beta distribution with parameters $\alpha=n+1,\beta=m+1$.
let us considers a genomic position in two individuals, and the corresponding methylation probabilities $X_1,X_2$. To explore whether the same position is differentially methylated across the individuals one
should compute $P(X_1>X_2)$. Customarily, this problem is tackled by simpler approximation : a particular handy one consists in performing a Fisher test over a contingency table built with the number of non converted and converted reads for individuals 1 and 2.
The fisher  test does not know aout the underlying beta distribution, hence one would expect some loss of power: here below I show that this is the case
\end{document}