\documentclass[11pt]{amsart}
\usepackage{eulervm}
\usepackage{graphicx}
\renewcommand{\rmdefault}{pplx}
\title{A note on exact differences between beta distributions in genomic (methylation) studies}
\author{Emanuele Raineri, Marc Dabad,Simon Heath}
\email{emanuele.raineri@gmail.com}
\date{\today}
\begin{document}
\begin{abstract}
We apply a known (\cite{exactbetaineq},\cite{numineq}) algorithm for computing exactly the difference between beta distributions
to compute the probability that a given position in a genome is differentially methylated across
samples. We discuss the advantages brought by the adoption of the exact solution
with respect to two approximations (Fisher's test and Z score).
The same formalism presented here can be applied in a similar way to variant calling.
\end{abstract}
\maketitle
\section{Introduction}
\subsection{Beta distribution to model methylation probabilities}
The Beta probability distribution (defined explicitly below)  appears very naturally in many studies of genomic data : tipically such analyses also entail the comparison between different samples, which in turn means that different Betas have to be combined. Here for concreteness we will describe the case of measuring DNA methylation through whole genome bisulfite sequencing (WGBS in what follows), but the same formalism  applies with almost no change to SNP calling. 

The general problem  of measuring DNA methylation differences between samples can be described for our purposes as follows : imagine two populations of cells. At any given genomic coordinate, a certain fraction of the DNA strands in the first population population will be methylated: we will indicate this quantity with $\theta_1$. The corresponding number for the second population is $\theta_2$. The purpose of WGBS is to estimate the $\theta$ of a given input population by measuring the methylation state of a random ( i.e. selected in some unpredictable way )  subset of it. The purpose of the software we will discuss in this note is to estimate $P(\theta_1>\theta_2)$ given the result of a WGBS experiment.

In what follows we will explain one way of building a probabilisic model of the above experiment, and study some features of an algorithm for exact comparison of Beta distributions. In particular we will show the advantages of the exact computation over two sensible approximations to it, namely performing a Fisher's test on the counts and computing a Z score test (thereby replacing the Beta with a Gaussian). An implementation of the exact algorithm is available on the web (see below).

Now on to some notation. We will indicate the Beta probability distribution (over $\theta$) with parameters $a,b$ with $\mbox{Beta}(a,b)$. On the other hand, we'll use the letter $B$ for the Beta function 
\[B(a,b)=\frac{(a-1)!(b-1)!}{(a+b-1)!}\] The $B$ function plays a role in the definition of the Beta distribution \[\mbox{Beta}(a,b)=\frac{\theta^{a-1}(1-\theta)^{b-1}}{B(a,b)}\]
 
Consider a set of reads out of a WGBS experiment covering a certain genomic coordinate $x$ with read depth $d$. Since not all cells in the sample being sequenced will, in general,  have the same basis methylated at the same time, the final result of a bisulfite experiment will be a collection of heterogeneous reads : some will indicate methylation at position $x$ (these are the so called {\em non converted} reads), others (the {\em converted} reads) will correspond to samples that are not methylated. What one wants to estimate is the probability $\theta$ of methylation in the population (as opposed to the sampled reads) at position $x$.  If $\theta$ is known a priori, the probability of obtaining $n$ non converted reads is :
\[P(n|\theta)={d \choose n}\theta^n ({1-\theta})^{(d-n)}=\frac{1}{d+1}\mbox{Beta}(n+1,d-n+1)\]
If one assumes a uniform prior on $\theta$, $P(\theta)=1 \ \forall \theta \in [0,1]$ the expression for $P(\theta|n)$ is very similar \footnote{The factor $\frac{1}{d+1}$ cancels out when applying Bayes' theorem}
\[P(\theta|n)=\mbox{Beta}(n+1,d-n+1)\]
Therefore, to assess whether a position is differentially methylated across two samples with non converted reads respectively $n_1,n_2$ and read depths $d_1,d_2$ one
has to compute 
\[P(\theta_1>\theta_2) \] where 
\[ \theta_1 \sim \mbox{Beta}(n_1+1,d_1-n_1+1) , \theta_2 \sim \mbox{Beta}(n_2+1,d_2-n_2+1))\]\label{ineq}

In the subsection below, I will outline the exact solution to \ref{ineq}.
 
\subsection{Exact computation of beta differences}
Let $I_x(a,b)$ the cumulative distribution function of a $Beta(a,b)$ distribution. Let also $g(a,b,c,d)=P(\theta_1>\theta_2)$ where $\theta_1$ and $\theta_2$ are distributed respectively as $Beta(a,b)$ and $Beta(c,d)$. The cumulative distribution function of a Beta distribution is called the incomplete Beta distribution, and I will use the notation $I_x(a,b)$ for it. Now, by definition one has \[P(\theta_1>\theta_2)=g(a,b,c,d)=\int_{-\infty}^{+\infty} Beta_x(a,b)I_x(c,d) dx\]
$g(a,b,c,d)$ possess a number of symmetries.
Using the identity \[I_x(c,d)=\frac{1}{cB(c,d)}x^c(1-x)^d+I_x(c+1,d)\] one finds that 
$g(a,b,c,d)=\frac{1}{c}h+g(a,b,c+1,d)$
\begin{align}
g(a + 1,b,c,d) &= g(a,b,c,d) + h(a,b,c,d)/a \\
g(a,b+1,c,d) &= g(a,b,c,d) - h(a,b,c,d)/b \\
g(a,b,c+1,d) &= g(a,b,c,d) - h(a,b,c,d)/c \\
g(a,b,c,d+1) &= g(a,b,c,d) + h(a,b,c,d)/d 
\end{align}
where \[h=\frac{B(a+c,b+d)}{B(a,b)B(c,d)}\]
\subsection{Approximate computation}
\section{Results and methods} 
\subsection{comparison with approximate results}
We organized the comparison between the exact and approximate solution in two steps. First,
we looked at the behaviour of the two tests on a pair of real samples. The results are shown in \ref{cmpreal}.
\begin{figure}[h]
\caption{comparing beta distribution with Fisher and Z score}
\includegraphics[width=\textwidth]{fig1}
\end{figure}\label{cmpreal}
on the $x$ axis we plotted $P(\theta_1>\theta_2)$, on the $y$ axis we plotted the corresponding $p$-value obtained by approximating the Beta respectively with a Fisher's test (on the left) and with a Gaussian (on the right).
We did the comparson over 100000 positions across two samples : the plot is in fact a density plot, in which different shades of blue indicate how many times the two values fall into a certain region of the plane. There's not much to comment there, except to note that, as expected, there's a broad correspondence between the different methods.

Next, we simulated a pair of samples whose counts are generated by the same underlying binomial process at different coverages, i.e. $\theta_1=\theta_2=0.5$. These consitute a negative control, in the sense that all the different methods should not report significant differences between the samples. Furthermore, I generated a pair of samples such that their underlying binomial probabilities are markedly different $\theta_1=0.9,\theta_2=0.5$; those are the true positives, samples for which the test should detect that $\theta_1>\theta_2$. I can then compare the ROC curves of the three methods for different values of the samples' coverages, $d_1,d_2$. 
The results are depicted infig \ref{roc}.
\begin{figure}[h]
\caption{ROC curves for the three methods under comparison. Each point in the ROC curve is obtained by choosing a different threshold for calling differential methylation. For the Z score test and the Fisher's test the $p$ values are:. For the Beta distributions the threshold probabilities are: }
\includegraphics[width=\textwidth]{fig2}
\end{figure}\label{roc}
\subsection{differentially methylated regions}
\subsection{implementation}
the algorithm described above is implemented in a \verb=C= program, called \verb=methyl_diff=, available from the github page of one of the authors : \verb=http://emanueleraineri.github.io/=. The program takes as input (from \verb=stdin=) four integers, i.e. the number of non converted and converted reads for the first and the second sample respectively, and prints $P(\theta_1>\theta_2)$ on the \verb=stdout=. It takes $3.3s$ to process $10^5$ lines on off-the-shelf hardware (MacBookPro with Intel \verb=i7@2.66GHz=) 
\bibliography{beta_paper}
\bibliographystyle{plain}
\end{document}
